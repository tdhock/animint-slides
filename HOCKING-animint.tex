\documentclass{beamer}

\usepackage{listings}
\usepackage{slashbox}
\usepackage{tikz}
\usepackage{booktabs}
\usepackage{amsmath,amssymb}
\usepackage{hyperref}
\usepackage{graphicx}

\DeclareMathOperator*{\argmin}{arg\,min}
\DeclareMathOperator*{\argmax}{arg\,max}
\DeclareMathOperator*{\maximize}{maximize}
\DeclareMathOperator*{\minimize}{minimize}
\newcommand{\sign}{\operatorname{sign}}
\newcommand{\RR}{\mathbb R}
\newcommand{\NN}{\mathbb N}

% Set transparency of non-highlighted sections in the table of
% contents slide.
\setbeamertemplate{section in toc shaded}[default][100]
\AtBeginSection[]
{
  \setbeamercolor{section in toc}{fg=red} 
  \setbeamercolor{section in toc shaded}{fg=black} 
  \begin{frame}
    \tableofcontents[currentsection]
  \end{frame}
}

\begin{document}

\title{Interactive, animated web graphics using the animint package}
\author{
Toby Dylan Hocking\\
joint work with Susan VanderPlas
}

\date{28 January 2014}

\maketitle

\begin{frame}
  \frametitle{animint renders interactive, animated web graphics}
  \url{http://github.com/tdhock/animint}
  \begin{itemize}
  \item Motivation: high-dimensional time series (several linked plots).
  \item User writes only R code (a list of ggplots).
  \item animint exports/converts to csv/json/html/Javascript files.
  \item Rendered in a web browser using the D3 Javascript library.\\
    Bostock et al 2011. D3 data-driven documents. IEEE Transactions on
    Visualization and Computer Graphics.
  \end{itemize}
\end{frame}

\begin{frame}
  \frametitle{animint adds clicksSelects and showSelected aesthetics
    to ggplot2}
  ggplot2 aesthetics map data variables to visual properties:
  \begin{description}
  \item[x] horizontal position.
  \item[y] vertical position.
  \item[color] of lines/points.
  \item[size] thickness/width of lines/points.
  \item[showSelected] only show the selected data subset.
  \item[clickSelects] clicking changes the selected value.
  \end{description}
\end{frame}

\begin{frame}[fragile]
  \frametitle{Demo code: a named list of ggplots with clickSelects and
    showSelected aesthetics}
\begin{verbatim}
library(animint)
TimeSeries <- ggplot()+
  geom_line(aes(showSelected=variable))+
  ...
Scatter <- ggplot()+
  geom_point(aes(clickSelects=variable))+
  ...
PlotList <- list(series=TimeSeries, scatter=Scatter)
gg2animint(PlotList)
\end{verbatim}
\end{frame}

\begin{frame}
  \frametitle{Comparison of packages that can make a multi-plot
    animation of WorldBank/Gapminder data}
\url{https://github.com/tdhock/interactive-tutorial/tree/master/animation}
%\url{http://sugiyama-www.cs.titech.ac.jp/~toby/animint/WorldBank/viz.html}
\vskip 1cm
  \begin{tabular}{ccccc}
    Package & years & interaction vars & programming & lines of R code \\
    \hline
    tcltk & 1991- & several & object & 100s \\
    animation & 2007- & 1 = time & imperative & 40 \\
    shiny & 2012- & several & reactive & 60 \\
    animint & 2013- & several & declarative & 20 
  \end{tabular}
  \vskip 1cm
  Points in common:
  \begin{itemize}
  \item Click to show/hide subsets of data.
  \item Animation.
  \item Multi-layer (e.g. points and lines).
  \item Write only R code (not JavaScript).
  \end{itemize}
\end{frame}

\begin{frame}
  \frametitle{Limitations, future work}
  \begin{itemize}
  \item Only interactivity is show/hide data subsets.
  \item Does not support zooming, brushing multiple data, annotation.
  \item Large data sets are slow (when $>10,000$ points onscreen).
  \end{itemize}
\end{frame}

\begin{frame}
  \frametitle{Thank you!}
  Supplementary slides appear after this one.
\end{frame}

\begin{frame}
  \frametitle{Google Summer of Code}
Student gets \$5000 for writing open source code for
    3 months.
    \begin{description}
    \item[March] Admins (such as myself) for open source organizations
      e.g. R, Bioconductor apply to Google.
    \item[April] Mentors suggest projects for each org.
    \item[May] Students submit project proposals to Google.
    \item Mentors rank student/project proposals.
    \item Google gives $n$ students to an org.
    \item[June]The top $n$ students get \$500 and begin coding.
    \item[August] Midterm evaluation, pass = \$2250.
    \item[September] Final evaluation, pass = \$2250.
    \item[November] Orgs get \$500/student mentored.
    \end{description}
    We can be \textbf{mentors} (get code written) and
    \textbf{students} (get paid).
\end{frame}

\end{document}
